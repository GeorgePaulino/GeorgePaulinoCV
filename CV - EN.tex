\documentclass{resume} % Use the custom resume.cls style

\usepackage[left=0.4 in,top=0.4in,right=0.4 in,bottom=0.4in]{geometry} % Document margins
\usepackage{hyperref}
\newcommand{\tab}[1]{\hspace{.2667\textwidth}\rlap{#1}} 
\newcommand{\itab}[1]{\hspace{0em}\rlap{#1}}
\name{George P. de Souza Filho}
\address{+55 (83) 9 86042068 - PB}
\address{george7paulino@gmail.com}

\begin{document}

\textbf{Languages}: Native Portuguese, Intermediate English.

\textbf{Github}: \href{https://github.com/GeorgePaulino}{GeorgePaulino} \ 
\textbf{Linkedin}:  \href{https://www.linkedin.com/in/georgepaulino/}{in/georgepaulino}

%----------------------------------------------------------------------------------------
%	EXPERIÊNCIA
%----------------------------------------------------------------------------------------


% \begin{rSection}{EXPERIÊNCIAS}

% \end{rSection} 

%----------------------------------------------------------------------------------------
%	EDUCAÇÃO
%----------------------------------------------------------------------------------------

\begin{rSection}{Education}
\item \textbf{\href{https://www.ufpb.br/}{Bachelor's Degree in Computer Engineering - Universidade Federal da Paraíba}} \hfill {08/2022 - 12/2027 (Expected)}
\item \textbf{\href{https://www.ifpb.edu.br/}{Multimedia Technician - Instituto Federal da Paraíba}} \hfill {02/2019 - 12/2021}
\end{rSection}


%----------------------------------------------------------------------------------------
% SKILLS	
%----------------------------------------------------------------------------------------

\begin{rSection}{SKILLS}

\item \begin{tabular}{ @{} >{\bfseries}l @{\hspace{6ex}} l }

Languages & C, C++, C\#, Java, JavaScript, Python, SQL. \\
Technologies \\ % & Database, Nodejs, PyTorch, React, Transformers. \
\quad Mobile & React Native, Xamarin. \\
\quad Web & NextJS, NodeJS, React. \\
\quad AI & Datasets, PyTorch, Transformers. \\
\quad Database & Sqlite, Sql Server. \\
Tools & Git, Latex, Linux, Windows. \\
\end{tabular}

\end{rSection}

%----------------------------------------------------------------------------------------
%	EXTRACURRICULAR
%----------------------------------------------------------------------------------------
\begin{rSection}{Extracurricular} \itemsep -1pt {}

    \item \textbf{\href{https://www.instagram.com/tailufpb/}{Member - TAIL}} \hfill 09/2022 - Present \\
    Participating in TAIL (Technology and Artificial Intelligence League), a study league at UFPB, working on projects involving AI. These projects aim to develop the knowledge of the members for solving real-world problems.
    
    \item \textbf{\href{https://www.escolainvictus.com/}{Robotics Assistant - Invictus Colégio e Curso}} \hfill 02/2022 - 08/2022 \\
    Invictus School hired a group of teachers to teach programming at the institute, and I was invited to assist in the programming part. I had the opportunity to work as a teacher due to my proficiency in the subject.
    
    %\item \textbf{Monitor - 1st Municipal Robotic Tournament, Cabedelo, PB} \hfill 11/2019 \\
    %Due to my performance in the National stage of the International Robotics Tournament (2017), I was invited to become a monitor at the city tournament. There, I was able to assess the performance of other students.
\end{rSection}

%----------------------------------------------------------------------------------------
%	PROJETOS
%----------------------------------------------------------------------------------------

\begin{rSection}{Projects}

\item \textbf{\href{https://enetrix.ufpb.br/}{Enetrix - UFPB - Website}} \hfill 2022 \\
A project of interest to the United Nations regarding international and national energy security agreements. Developed by UFPB, the project was featured on the \href{https://www.un.org/en/academic-impact/insights-energy-diplomacy-and-un-data-intensive-research-brazilian-universities}{United Nations} portal. I was responsible for referencing articles and developing the official website.

\item \textbf{\href{https://github.com/ralfferreira/generate-abstract}{IAbstract - TAIL - AI}} \hfill 2022 \\
Natural Language Processing (NLP) artificial intelligence that has the ability to summarize academic articles. This project was developed during my first semester in college as part of TAIL. I assisted in developing metrics and training two different models used in the project.

\item \textbf{\href{https://georgepaulino.github.io/TicTacToe/}{Tic Tac Toe - Independent - Website}} \hfill 2022 \\
A website to play Tic Tac Toe developed using React JavaScript with TypeScript. I independently worked on this project to study React and its technologies.

\end{rSection}

%----------------------------------------------------------------------------------------
%	CONQUISTAS
%----------------------------------------------------------------------------------------

\begin{rSection}{Achievements}

\item \textbf{\href{http://www.obmep.org.br/}{Multimedalist - Brazilian Mathematics Olympiad (OBMEP)}} \hfill 2021 \
OBMEP is renowned for the difficulty of its exams held throughout Brazil. I have received two bronze medals (2015, 2021) and one silver medal (2018) in the competition.

\item \textbf{\href{http://www.mat.ufpb.br/opm/}{Multimedalist - Olympiad of Mathematics of João Pessoa (OPM)}} \hfill 2019 \
OPM is a regional mathematics olympiad held in Paraíba. It presents challenging problems, and I have earned a bronze medal (2018) and a silver medal (2019) in my participation in the exams.

\end{rSection}

\end{document}
