\documentclass{resume} % Use the custom resume.cls style

\usepackage[left=0.4 in,top=0.4in,right=0.4 in,bottom=0.4in]{geometry} % Document margins
\usepackage{hyperref}
\newcommand{\tab}[1]{\hspace{.2667\textwidth}\rlap{#1}} 
\newcommand{\itab}[1]{\hspace{0em}\rlap{#1}}
\name{George P. de Souza Filho} % Your name
\address{+55 83 986042068 – george7paulino@gmail.com – BR PB} % Phone, Email, State
%\address{ Cabedelo, Paraíba, Brasil. } % Your secondary address

\begin{document}
	
	
\begin{tabular}{ @{} >{\bfseries}l @{\hspace{6ex}} l }
	Languages & Native Portuguese, Intermediary English.
\end{tabular}
	
\textbf{Github} : \href{https://github.com/GeorgePaulino}{GeorgePaulino} \ \textbf{Linkedin} :  \href{https://www.linkedin.com/in/georgepaulino/}{in/georgepaulino}
	
%----------------------------------------------------------------------------------------
%	EXPERIENCES
%----------------------------------------------------------------------------------------

\begin{rSection}{EXPERIENCES}
	
	\item \textbf{Assistant in Robotics Classes - \href{https://www.escolainvictus.com/}{Invictus}} \hfill 02/2022 - 08/2022 \\
	Invictus School hired a group of teachers to teach robotic classes at the institute, I was invited by them to help with the programming part. Giving classes by many times.
	
\end{rSection} 

%----------------------------------------------------------------------------------------
%	EDUCATION
%----------------------------------------------------------------------------------------

\begin{rSection}{Education}
	
	\item \textbf{Graduation in Computer Engineering - UFPB } \hfill {08/2022 - 12/2027 (expected)}
	
	\item \textbf{Multimedia Technician - IFPB} \hfill {02/2019 - 12/2021}
	
\end{rSection}

%----------------------------------------------------------------------------------------
% SKILLS	
%----------------------------------------------------------------------------------------

\begin{rSection}{Skills}
	
	\item \begin{tabular}{ @{} >{\bfseries}l @{\hspace{6ex}} l }
		
		Languages & C, C++, C\#, Dart, Java, JavaScript, Python, SQL. \\
		Tools & .NET, Electron, Flutter, Nodejs, React, Sql Serve, Typescript. \\
		%Web & React, Typescript, Blazor, Nodejs. \\
		%Apps & Electron, Xamarin. \\
		%Databases & Sqlite, Sql Serve. \\
		Technologies & Git, Latex, Linux. \\
		Interests & Front-end, Back-end, Software Development. \\
	\end{tabular}
	
\end{rSection}

%----------------------------------------------------------------------------------------
%	PROJECTS
%----------------------------------------------------------------------------------------

\begin{rSection}{Projects}
	
	\item \textbf{\href{https://github.com/GeorgePaulino/PingPongLan}{Ping Pong Lan}} \\
	Ping Pong desktop application, allows connection via lan and aimed to test the Monogame graphics library.
	
	\item \textbf{\href{https://georgepaulino.github.io/TicTacToe/}{Tic Tac Toe}} \\
	Website to play the tic-tac-toe game. Developed in React TypeScript.
	
\end{rSection} 

%----------------------------------------------------------------------------------------
%	EXTRACURRICULAR
%----------------------------------------------------------------------------------------

\begin{rSection}{Extracurricular} \itemsep -1pt {}
	
	\item \textbf{Integral – \href{https://www.instagram.com/tailufpb/}{TAIL} – UFPB} \hfill 09/2022 \\
	Working at TAIL (Technology and Artificial Intelligence League) in the development of projects using AI for resolution of real problems.
	
	\item \textbf{Monitor – 1º Torneio Municipal de Robotrônica, Cabedelo, PB} \hfill 11/2019 \\
	Thanks to my performance in the National stage of the International Robotics Tournament (2017), I was invited to become a monitor at the city's tournament. There, I was able to punctuate the performance of other students. 
\end{rSection}

%----------------------------------------------------------------------------------------
%	PUBLICATIONS
%----------------------------------------------------------------------------------------

%\begin{rSection}{PUBLISHED PAPERS} \itemsep -1pt {}
%\item \textbf{} \hfill 
%\end{rSection}

%----------------------------------------------------------------------------------------
%	CERTIFICATES
%----------------------------------------------------------------------------------------

% \begin{rSection}{RELEVANT CERTIFICATES}
% \item \textbf{} \hfill 09/2020
% \end{rSection}

%----------------------------------------------------------------------------------------
%	ACHIEVEMENTS
%----------------------------------------------------------------------------------------

\begin{rSection}{Achievements}
	
	\item \textbf{Multi Medalist –  \href{http://www.obmep.org.br/}{OBMEP}} \hfill 2021 \\
	The Brazilian Mathematics Olympiad (OBMEP) is renowned for the difficulty it presents in its tests applied throughout Brazil. I have conquered two bronze medal awards (2015, 2021) and one silver medal (2018) in the test of this olympic.
	
	\item \textbf{Multi Medalist – \href{http://www.mat.ufpb.br/opm/}{OPM}} \hfill 2019 \\
	The Pessoense Mathematics Olympiad (OPM) is a regional Olympiad developed in Paraíba. It is challenging, even so, I got a Bronze (2018) and Silver (2019) medals in my performance in the tests.
	
\end{rSection}

\end{document}
