\documentclass{resume} % Use the custom resume.cls style

\usepackage[left=0.4 in,top=0.4in,right=0.4 in,bottom=0.4in]{geometry} % Document margins
\usepackage{hyperref}
\newcommand{\tab}[1]{\hspace{.2667\textwidth}\rlap{#1}} 
\newcommand{\itab}[1]{\hspace{0em}\rlap{#1}}
\name{George P. de Souza Filho}
\address{+55 (83) 9 86042068 - PB}
\address{george7paulino@gmail.com}

\begin{document}

\textbf{Linguagens}: Português Nativos, Inglês Intermediário.

\textbf{Github}: \href{https://github.com/GeorgePaulino}{GeorgePaulino} \ 
\textbf{Linkedin}:  \href{https://www.linkedin.com/in/georgepaulino/}{in/georgepaulino}

%----------------------------------------------------------------------------------------
%	EXPERIÊNCIA
%----------------------------------------------------------------------------------------


% \begin{rSection}{EXPERIÊNCIAS}

% \end{rSection} 

%----------------------------------------------------------------------------------------
%	EDUCAÇÃO
%----------------------------------------------------------------------------------------

\begin{rSection}{Educação}

\item \textbf{\href{https://www.ufpb.br/}{Graduação em Engenharia da Computação - Universidade Federal da Paraíba}} \hfill {08/2022 - 12/2027 (Esperado)}

\item \textbf{\href{https://www.ifpb.edu.br/}{Técnico em Multimídia - Instituto Federal da Paraíba}} \hfill {02/2019 - 12/2021}

\end{rSection}


%----------------------------------------------------------------------------------------
% HABILIDADES	
%----------------------------------------------------------------------------------------
\begin{rSection}{HABILIDADES}

\item \begin{tabular}{ @{} >{\bfseries}l @{\hspace{6ex}} l }

Linguagens & C, C++, C\#, Java, JavaScript, Python, SQL. \\
Tecnologias \\ % & Database, Nodejs, PyTorch, React, Transformers. \\
\quad Mobile & React Native, Xamarin. \\
\quad Web & NextJS, NodeJS, React. \\
\quad IA & Datasets, PyTorch, Transformers. \\
\quad Database & Sqlite, Sql Serve. \\
Ferramentas & Git, Latex, Linux, Windows. \\
\end{tabular}

\end{rSection}

%----------------------------------------------------------------------------------------
%	EXTRACURRICULAR
%----------------------------------------------------------------------------------------
\begin{rSection}{Extracurriculares} \itemsep -1pt {}

    \item \textbf{\href{https://www.instagram.com/tailufpb/}{Integrante – TAIL}} \hfill 09/2022 - Atualidade \\
    Participando da TAIL (Technology and Artificial Intelligence League), liga de estudos da UFPB, atuando no desenvolvimento de projetos utilizando IA. Projetos estes que buscam desenvolver o conhecimento dos integrantes para resolução de problemas reais.
    
    \item \textbf{\href{https://www.escolainvictus.com/}{Auxiliar em Robótica – Invictus Colégio e Curso}} \hfill 02/2022 - 08/2022 \\
    O colégio Invictus contratou um grupo de professores para dar aulas de programação no instituto, fui convidado por eles á auxiliar na parte da programação. Acabei tendo a oportunidade de atuar como professor por ter facilidade no assunto.
    
    %\item \textbf{Monitor – 1º Torneio Municipal de Robotrônica, Cabedelo, PB} \hfill 11/2019 \\
    %Graças a minha atuação na etapa Nacional do Torneio Internacional de Robótica (2017), fui convidado a me tornar monitor no torneio da cidade. Lá, pude pontuar a atuação dos demais alunos. 
\end{rSection}

%----------------------------------------------------------------------------------------
%	PROJETOS
%----------------------------------------------------------------------------------------

\begin{rSection}{Projetos}

\item \textbf{\href{https://enetrix.ufpb.br/}{Enetrix – UFPB – Site}} \hfill 2022 \\
Projeto de interesse da ONU quanto a acordos internacionais e nacionais de segurança da energia. Desenvolvido pela UFPB, o projeto foi destacado no portal \href{https://www.un.org/en/academic-impact/insights-energy-diplomacy-and-un-data-intensive-research-brazilian-universities}{United Nations}. Responsável por referenciar artigos, bem como desenvolver o website oficial.

\item \textbf{\href{https://github.com/ralfferreira/generate-abstract}{IAbstract – TAIL – IA}} \hfill 2022 \\
Inteligência artificial de processamento de linguagem natural (NLP) que possui a capacidade de resumir artigos acadêmicos. Projeto desenvolvido durante 3 meses que integrei na TAIL no primeiro período da faculdade. Fui responsável por ajudar no desenvolvimento das métricas e treinamento de dois modelos diferentes utilizados no projeto. 

\item \textbf{\href{https://georgepaulino.github.io/TicTacToe/}{Tic Tac Toe – Independente – Site}}  \hfill 2022 \\
Website para jogar jogo da velha desenvolvido em React JavaScript com TypeScript. Projeto desenvolvido de maneira autônoma para o estudo do React e suas tecnologias.

\end{rSection} 

%----------------------------------------------------------------------------------------
%	CONQUISTAS
%----------------------------------------------------------------------------------------

\begin{rSection}{Conquistas}

\item \textbf{\href{http://www.obmep.org.br/}{Multimedalhista – Olimpíada Brasileira de Matemática (OBMEP)}} \hfill 2021 \\
A OBMEP é prestigiada pela dificuldade que apresenta em suas provas aplicadas no Brasil todo. Possuo duas premiações de bronze (2015, 2021) e uma de prata (2018) na prova.

\item \textbf{\href{http://www.mat.ufpb.br/opm/}{Multimedalhista – Olímpiada Pessoense de Matemática (OPM)}} \hfill 2019 \\
A OPM é uma olimpíada regional desenvolvida na Paraíba. Ela se mostra desafiadora, ainda assim, obtive uma medalha de Bronze (2018) e Prata (2019) em minha atuação nas provas.

\end{rSection}

\end{document}
