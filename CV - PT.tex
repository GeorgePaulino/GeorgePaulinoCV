\documentclass{resume} % Use the custom resume.cls style

\usepackage[left=0.4 in,top=0.4in,right=0.4 in,bottom=0.4in]{geometry} % Document margins
\usepackage{hyperref}
\newcommand{\tab}[1]{\hspace{.2667\textwidth}\rlap{#1}} 
\newcommand{\itab}[1]{\hspace{0em}\rlap{#1}}
\name{George P. de Souza Filho}
\address{+55 83 986042068 - george7paulino@gmail.com - PB}

\begin{document}


\begin{tabular}{ @{} >{\bfseries}l @{\hspace{6ex}} l }
Linguagens & Português Nativos, Inglês Intermediário.
\end{tabular}

\textbf{Github} : \href{https://github.com/GeorgePaulino}{GeorgePaulino} \ \textbf{Linkedin} :  \href{https://www.linkedin.com/in/georgepaulino/}{in/georgepaulino}

%----------------------------------------------------------------------------------------
%	EXPERIÊNCIA
%----------------------------------------------------------------------------------------


\begin{rSection}{EXPERIÊNCIAS}

\item \textbf{Auxiliar em Aulas de Robótica - \href{https://www.escolainvictus.com/}{Invictus}} \hfill 02/2022 - 08/2022 \\
O colégio Invictus contratou um grupo de professores para dar aulas de programação no instituto, fui convidado por eles á auxiliar na parte da programação. Acabei tendo a oportunidade de atuar como professor por ter facilidade no assunto.

\end{rSection} 

%----------------------------------------------------------------------------------------
%	EDUCAÇÃO
%----------------------------------------------------------------------------------------

\begin{rSection}{Educação}

\item \textbf{Graduação em Engenharia da Computação - Universidade Federal da Paraíba} \hfill {08/2022 - 12/2027 (expected)}

\item \textbf{Técnico em Multimídia - Instituto Federal da Paraíba} \hfill {02/2019 - 12/2021}

\end{rSection}


%----------------------------------------------------------------------------------------
% HABILIDADES	
%----------------------------------------------------------------------------------------
\begin{rSection}{HABILIDADES}

\item \begin{tabular}{ @{} >{\bfseries}l @{\hspace{6ex}} l }

Linguagens & C, C++, C\#, Java, JavaScript, Python, SQL. \\
Ferramentas & .NET, Electron, Nodejs, React, Sql Serve, Typescript. \\
%Web & React, Typescript, Blazor, Nodejs. \\
%Aplicativos & Electron, Xamarin. \\
%Banco de Dados & Sqlite, Sql Serve. \\
Tecnologias & Git, Latex, Linux. \\
Interesses & Front-end, Back-end, Desenvolvimento de Software. \\
\end{tabular}

\end{rSection}

%----------------------------------------------------------------------------------------
%	PROJETOS
%----------------------------------------------------------------------------------------

\begin{rSection}{Projetos}

\item \textbf{\href{https://github.com/GeorgePaulino/PingPongLan}{Ping Pong}} \\
Aplicativo de desktop de Ping Pong, permiti conexão via lan e tinha como objetivo testar a biblioteca gráfica Monogame.

\item \textbf{\href{https://georgepaulino.github.io/TicTacToe/}{Tic Tac Toe}} \\
Website para jogar jogo da velha desenvolvido em React TypeScript.

\end{rSection} 

%----------------------------------------------------------------------------------------
%	EXTRACURRICULAR
%----------------------------------------------------------------------------------------
\begin{rSection}{Extracurriculares} \itemsep -1pt {}

\item \textbf{Integrante – \href{https://www.instagram.com/tailufpb/}{TAIL} – UFPB} \hfill 09/2022 \\
Atuando na TAIL (Technology and Artificial Intelligence League) no desenvolvimento de projetos utilizando IA. Estes buscam desenvolver o conhecimento dos integrantes para resolução de problemas reais.

\item \textbf{Monitor – 1º Torneio Municipal de Robotrônica, Cabedelo, PB} \hfill 11/2019 \\
Graças a minha atuação na etapa Nacional do Torneio Internacional de Robótica (2017), fui convidado a me tornar monitor no torneio da cidade. Lá, pude pontuar a atuação dos demais alunos. 
\end{rSection}

%----------------------------------------------------------------------------------------
%	CONQUISTAS
%----------------------------------------------------------------------------------------

\begin{rSection}{Conquistas}

\item \textbf{\href{http://www.obmep.org.br/}{Multimedalhista – Olimpíada Brasileira de Matemática (OBMEP)}} \hfill 2021 \\
A OBMEP é prestigiada pela dificuldade que apresenta em suas provas aplicadas no Brasil todo. Possuo duas premiações de bronze (2015, 2021) e uma de prata (2018) na prova.

\item \textbf{\href{http://www.mat.ufpb.br/opm/}{Multimedalhista – Olímpiada Pessoense de Matemática (OPM)}} \hfill 2019 \\
A OPM é uma olimpíada regional desenvolvida na Paraíba. Ela se mostra desafiadora, ainda assim, obtive uma medalha de Bronze (2018) e Prata (2019) em minha atuação nas provas.

\end{rSection}

\end{document}
